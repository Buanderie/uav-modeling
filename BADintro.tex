\chapter{Introduction}

The Basic Air Data project (\url{www.basicairdata.eu}) brings together people (all three of us!) from different backgrounds and varying interests. Some are business owners and some are students. Some are mechanical engineers and some are electrical engineers. Well, we do have in common that we are all engineers. But the point is that we approach the problem of building useful air data measurement equipment from different perspectives.

Jose went ahead and built a complimentary website to present his Pitot-tube-building efforts, in hopes that he will draw like-minded individuals who will accompany him in his journey. Graziano did what a mechanical engineer does best - took a simple Pitot tube and produced a modular removable probe along with a killer mechanical drawing. Me, I like math and I like having things neat. During my research I often have to model dynamics, simulate objects and solve equations. However, when the set of your model equations reaches the order of hundreds I start to misplace them and having to re-start the bibliographical search to re-locate them and their source.

This document is the answer to my need to have all my aircraft equations and UAV-related models in one place, with reliable literature to support them. I has already been useful to me as a reference and as a quick source of models for my colleagues, and I hope you will find it equally useful.
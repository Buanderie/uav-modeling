\section{Wind Model}\label{sec:wind_model}

Wind models are mostly useful for validation of control schemes under simulated disturbances. This section extends the discussion of Section \ref{sec:wind}.

Usually we consider the vertical component of the wind to be zero ($v_{ws,d}=0$), but this is not true in environments such as mountain slopes or in regions with thermal currents.

A useful expression for constant wind is:
\begin{equation}
\bm{v_{ws}} = \bm{R}_b^TV_{ws}[-\cos\theta_w, -\sin\theta_w, 0]^T
\end{equation}
where $V_{ws}$ is the wind overall magnitude (in $m/s$) and $\theta_w$ is the wind direction (\SI{0}{\degree} for North wind)

\begin{lstlisting}[style=C-style]
u_ws = (cos(theta)*cos(psi))*(-cos(theta_wind)*V_ws) + (cos(theta)*sin(psi))*(-sin(theta_wind)*V_ws)
v_ws = (sin(phi)*sin(theta)*cos(psi)-cos(phi)*sin(psi))*(-cos(theta_wind)*V_ws) + (sin(phi)*sin(theta)*sin(psi)+cos(phi)*cos(psi))*(-sin(theta_wind)*V_ws)
w_ws = (cos(phi)*sin(theta)*cos(psi)+sin(phi)*sin(psi))*(-cos(theta_wind)*V_ws) + (cos(phi)*sin(theta)*sin(psi)-sin(phi)*cos(psi))*(-sin(theta_wind)*V_ws)
\end{lstlisting}

This allows us to introduce an altitude model for the wind magnitude, commonly known as the Power Law:
\begin{equation}
V_{ws} = V_{ws,h_r} \left(\frac{h}{h_r}\right)^\alpha
\end{equation}

\begin{lstlisting}[style=C-style]
V_ws = V_ws_ref*pow(h/h_ref,exp_Hell)
\end{lstlisting}

This formula describes the increase of the wind magnitude as an exponential function of the altitude, given a measurement of wind magnitude $V_{ws,h_r}$ at altitude $h_r$. $\alpha$ is the Hellmann exponent (roughness exponent), which describes the wind shear effect. Commonly, surface wind magnitude measurements are available for the altitude of 10m, so the above formula becomes
\begin{equation} \label{eq:staticWind10}
V_{ws} = V_{ws,h_{10}} \left(\frac{h}{h_{10}}\right)^\alpha
\end{equation}

Values for $\alpha$ vary with terrain morphology and wind turbulence. Some indicative values can be found in references \cite{Banuelos-Ruedas2011}, \cite{Peterson1978}, \cite{wiki:WindGrad}, \cite{wiki:Wind_profile_power_law}, but it is also claimed that the "1/7th power law" (using a value $\alpha=1/7$) is an adequate approximation for most purposes.
It should be noted, however, that most of the reviewed studies where oriented towards wind farm applications and wind models where validated up to a few hundred meters above ground.

The following table is pulled from \cite{Banuelos-Ruedas2011}.

\begin{table}[H]
	\centering
	\begin{tabular}{|c|c|}
		\hline
		Landscape type             & Friction coefficient $\alpha$ \\ \hline
		Lakes, ocean and smooth hard ground  &              0.1              \\ \hline
		Grasslands (ground level)       &             0.15              \\ \hline
		Tall crops, hedges and shrubs     &             0.20              \\ \hline
		Heavily forested land         &             0.25              \\ \hline
		Small town with some trees and shrubs &              0.3              \\ \hline
		City areas with high rise buildings  &              0.4              \\ \hline
	\end{tabular} 
	\caption{Hellmann Exponent Values over Various Terrain}
\end{table}

An alternative model for wind shear can be found in \cite{Moorhouse1982}.

Wind gusts can be modeled using Dryden transfer functions, as presented in \cite{Moorhouse1982}, \cite{BEAL1993}, \cite{MathWorks:DrydenTurbulence} and \cite{Beard2012} . Time responses can be generated by feeding unit variance white noise in the following filters. Note that airspeed is a parameter for these functions, but it can be replaced with the mean or cruise airspeed of the aircraft.
\begin{equation}
\bm{V_{wg}}(s) =
\begin{bmatrix}
\sigma_u \sqrt{\frac{2V_a}{\pi L_u}} \frac{1}{s + \frac{V_a}{L_u}}\\
\sigma_v \sqrt{\frac{3V_a}{\pi L_v}} \frac{s+\frac{V_a}{\sqrt{3}L_v}}{(s+\frac{V_a}{L_u})^2} \\
\sigma_w \sqrt{\frac{3V_a}{\pi L_w}} \frac{s+\frac{V_a}{\sqrt{3}L_w}}{(s+\frac{V_a}{L_w})^2}
\end{bmatrix}
\end{equation}

During discrete implementation of the transfer functions, since first and second order functions are used, past values of the wind gusts are maintained and used.
\begin{lstlisting}[style=C-style]
u_dg = w_dg_prev*(1 - V_a/L_u*dt) + sigma_u*sqrt(1*V_a/(pi*L_u))*dt*noise
V_dg[1] = -pow(V_a/L_u,2)*dt*V_dg_prev[0] + V_dg_[1] + sigma_u*sqrt(3*V_a/(pi*L_u))*V_a/(sqrt(3)*L_u)*dt*noise
V_dg[0] = (1 - 2*V_a/L_u*dt)*V_dg_prev[0] + dt*V_dg_prev[1] + sigma_u*sqrt(3*V_a/(pi*L_u))*dt*noise
v_dg = V_dg[0]
W_dg[1] = -pow(V_a/L_w,2)*dt*W_dg_prev[0] + W_dg_[1] + sigma_w*sqrt(3*V_a/(pi*L_w))*V_a/(sqrt(3)*L_w)*dt*noise
W_dg[0] = (1 - 2*V_a/L_w*dt)*W_dg_prev[0] + dt*W_dg_prev[1] + sigma_w*sqrt(3*V_a/(pi*L_w))*dt*noise	
w_dg = W_dg[0]
\end{lstlisting}

% Read BEAL1993 and verify the results

Typical values for the transfer function parameters for a small UAV can be found in \cite{Langelaan2011} and are copied below.
\begin{table}[H]
	\centering
	\begin{tabular}{|p{5cm}|c|c|c|c|c|}
		\hline
		Description              & altitude (m) & $L_u$ (m) & $L_w$  (m) & $\sigma _u$ (m/s) & $\sigma _w$ (m/s) \\ \hline
		low altitude, light turbulence    &      50      &    200    &     50     &       1.06       &       0.7        \\ \hline
		low altitude, moderate turbulence   &      50      &    200    &     50     &       2.12       &       1.4        \\ \hline
		medium altitude, light turbulence   &     600      &    533    &    533     &       1.5        &       1.5        \\ \hline
		medium altitude, moderate turbulence &     600      &    533    &    533     &       3.0        &       3.0        \\ \hline
	\end{tabular} 
	\caption{Gust Field Properties}
\end{table}
\section{Aerodynamics}
This is the defining part of the flight characteristics of an airplane. The aerodynamic response of the airplane is crucial for its stability as well as its maneuverability.
Aerodynamic forces are presented in two frames of reference. The first is the body frame, whose corresponding forces, $\bm{F_a}$ are used for kinematic calculations. 
\begin{equation}
	\bm{F_a} = \begin{bmatrix}
	F_{ax} \\ F_{ay} \\ F_{az}
	\end{bmatrix}
\end{equation}
Aerodynamic forces are considered to be exerted, under normal operation, at the \emph{center of lift} ($\bm{p}_{CoL}$), a point located on the symmetry plane of the aircraft at one-quarter of the main wing chord.

However, for both historical and practical reasons, the flight characteristics of the airplane and the resulting forces (lift, drag and sideforce) are expressed in the stability frame (see \ref{eq:StabMatrix}). Models of aircraft aerodynamics are also parametrized so as to predict the aerodynamic forces in the stability frame as well.

\begin{equation}
	\bm{F_s} = \begin{bmatrix}
	-F_D \\ F_Y \\ -F_L
	\end{bmatrix}
\end{equation}

Thus, we can obtain $\bm{F}_a$ via the equation
\begin{equation}
	\bm{F}_a = \bm{S}^T \bm{F}_s
\end{equation}

We remind that
\begin{equation*}
	\bm{S} =
	\begin{bmatrix}
		\cos \alpha \cos \beta  & \sin \beta & \sin \alpha \cos \beta  \\
		-\cos \alpha \sin \beta & \cos \beta & -\sin \alpha \sin \beta \\
		-\sin \alpha            & 0          & \cos \alpha
	\end{bmatrix}
\end{equation*}

\begin{IEEEeqnarray}{rCl}
	F_{ax} &= & -\cos \alpha F_D -\cos \alpha \sin \beta F_Y + \sin \alpha F_L \IEEEyessubnumber\\
	F_{ay} &= & -\sin \beta F_D + \cos \beta F_Y \IEEEyessubnumber \\
	F_{az} &= & -\sin \alpha \cos \beta F_D -\sin \alpha \sin \beta F_Y - \cos \alpha F_L \IEEEyessubnumber
\end{IEEEeqnarray}

\begin{lstlisting}[style=C-style]
	F_a_x = -cos(alpha)*F_D - cos(alpha)*sin(beta)*F_Y + sin(alpha)*F_L
	F_a_y = -sin(beta)*F_D + cos(beta)*F_Y
	F_a_z = -sin(alpha)*cos(beta)*F_D - sin(alpha)*sin(beta)*F_Y - cos(alpha)*F_L
\end{lstlisting}

Aerodynamic moments need not be modeled in the stability frame. However, forces are generally not applied on the center of gravity of the aircraft, thus we need to model the moment they apply on it separately.
Sources of such additional moments are the stability margin (how much forward than the center of lift lies the center of mass), height difference between center of mass and center of lift (for example in a high-wing aircraft) and mass additions and subtractions to the airframe.

\begin{equation}
	\bm{T_{a,tot}} = \begin{bmatrix}
		T_{ax} \\ T_{ay} \\ T_{az}
		\end{bmatrix}
		+\left(\bm{p}_{CoL} - \bm{p}_{CG}\right) \times \bm{F}_a
\end{equation}
\begin{IEEEeqnarray}{rCl}
	T_{ax,tot} &= & T_{ax} -dz F_{ay} + dy F_{az}\IEEEyessubnumber\\
	T_{ay,tot} &= & T_{ay} +dz F_{ax} - dx F_{az}\IEEEyessubnumber\\
	T_{az,tot} &= & T_{az} -dy F_{ax} + dx F_{ay}\IEEEyessubnumber
\end{IEEEeqnarray}

\begin{lstlisting}[style=C-style]
	dx = p_cl_x - p_cm_x
	dy = p_cl_y - p_cm_y
	dz = p_cl_z - p_cm_z
	T_atot_x = T_a_x - dz*F_a_y + dy*F_a_z
	T_atot_y = T_a_y + dz*F_a_x - dx*F_a_z
	T_atot_z = T_a_z - dy*F_a_x + dx*F_a_y
\end{lstlisting}

The next step in exploring the aerodynamic model is the definition of dynamic pressure. This quantity expresses the pressure exerted by the air moving around the airplane per unit of surface. It is proportional with the density of the air and with the square of the airspeed. Dynamic pressure is a crucial scaling factor for the rest of the aerodynamic equations
\begin{equation}
	\bar{q} = \frac{1}{2}\rho V_a^2
\end{equation}

\begin{lstlisting}
	q_bar = 0.5*rho*V_a*V_a
\end{lstlisting}

The air density, $\rho$, is dependent upon altitude. Its sea level value ($\rho_0 = 1.225 kg/m^3$) is often used as an approximate value for low-altitude flight, but if needed, accurate atmospheric models are provided in Section \ref{sec:atm_model}.

We are now able to express the force components as the product of dynamic pressure, the wing surface $S$, and the aerodynamic coefficients of lift, drag and sideforce, $C_L$, $C_D$, $C_Y$. It is very important to appreciate that these coefficients are not constant values; instead, they are functions of the state of the flying body, albeit their dependence from some state variables is more dominant than others. A more detailed explanation is given further down.
\begin{IEEEeqnarray}{rCl}
	F_D &= &\bar{q} S C_D \IEEEyesnumber \IEEEyessubnumber \\
	F_Y &= &\bar{q} S C_Y \IEEEyessubnumber\\
	F_L &= &\bar{q} S C_L \IEEEyessubnumber
\end{IEEEeqnarray}

\begin{lstlisting}[style=C-style]
	F_D = q_bar*S*C_D
	F_Y = q_bar*S*C_Y
	F_L = q_bar*S*C_L
\end{lstlisting}

Generally, the aerodynamic coefficients are functions of the airspeed, aerodynamic angles, angular velocity and control inputs.

\begin{IEEEeqnarray}{rCl}
	C_D &= &C_D(V_a, \alpha, q, \delta_e) \IEEEyesnumber \IEEEyessubnumber \\
	C_Y &= &C_Y(V_a, \beta, p, r, \delta_a, \delta_r) \IEEEyessubnumber\\
	C_L &= &C_L(V_a, \alpha, q, \delta_e) \IEEEyessubnumber
\end{IEEEeqnarray}

The aerodynamic moments are defined in a similar fashion, with the additional inclusion of wingspan $b$ and wing chord $c$. The notation for the moment coefficients may vary from source to source.
\begin{IEEEeqnarray}{rCl}
	T_{ax} &= &\bar{q} S b C_l \IEEEyesnumber \IEEEyessubnumber \\
	T_{ay} &= &\bar{q} S c C_m \IEEEyessubnumber\\
	T_{az} &= &\bar{q} S b C_n \IEEEyessubnumber
\end{IEEEeqnarray}

\begin{lstlisting}[style=C-style]
	T_a_x = q_bar*S*b*C_l
	T_a_y = q_bar*S*c*C_m
	T_a_z = q_bar*S*b*C_n
\end{lstlisting}
%
\begin{IEEEeqnarray}{rCl}
	C_l &= &C_l(V_a, \beta, p, r, \delta_a, \delta_r) \IEEEyesnumber \IEEEyessubnumber \\
	C_m &= &C_m(V_a, \alpha, q, \delta_e) \IEEEyessubnumber\\
	C_n &= &C_n(V_a, \beta, p, r, \delta_a, \delta_r) \IEEEyessubnumber
\end{IEEEeqnarray}
%%
Finally, a model for the aerodynamic coefficients is provided. Undoubtedly, the most common and established approach is to use multivariable polynomials which are linear with respect to their coefficients. This representation lends itself nicely for parameter identification techniques, such as least-squares flavours, very common in real-world aircraft design and testing.
%
\begin{IEEEeqnarray}{rCl} \label{eq:forceCoeff} 
	C_D &=& C_{D,0} + C_{D,\alpha}\alpha+C_{D,q}\frac{c}{2V_a}q + C_{D,\delta_e}\delta_e \IEEEyesnumber \IEEEyessubnumber\\
	C_Y &=& C_{Y,0} + C_{Y,\beta}\beta + C_{Y,p}\frac{b}{2V_a}p + C_{Y,r}\frac{b}{2V_a}r + C_{Y,\delta_a}\delta_a + C_{Y,\delta_r}\delta_r \IEEEyessubnumber\\
	C_L &=& C_{L,0} + C_{L,\alpha}\alpha+C_{L,q}\frac{c}{2V_a}q +  C_{L,\delta_e}\delta_e \IEEEyessubnumber	
\end{IEEEeqnarray}
%
\begin{lstlisting}
C_D = C_D_0 + C_D_alpha*alpha + C_D_q*c*0.5/V_a*q + C_D_delta_e*delta_e
C_Y = C_Y_0 + C_Y_beta*beta + C_Y_p*b*0.5/V_a*p + C_Y_r*b*0.5/V_a*r + C_Y_delta_a*delta_a + C_Y_delta_e*delta_e
C_L = C_L_0 + C_L_alpha*alpha + C_L_q*c*0.5/V_a*q + C_L_delta_e*delta_e
\end{lstlisting}
%
\begin{IEEEeqnarray}{rCl} \label{eq:torqueCoeff} 
	C_l &=& C_{l,0} + C_{l,\beta}\beta + C_{l,p}\frac{b}{2V_a}p + C_{l,r}\frac{b}{2V_a}r + C_{l,\delta_a}\delta_a + C_{l,\delta_r}\delta_r \IEEEyesnumber \IEEEyessubnumber \\
	C_m &=& C_{m,0} + C_{m,\alpha}\alpha + C_{m,q}\frac{c}{2V_a}q + C_{m,\delta_e}\delta_e \IEEEyessubnumber\\
	C_n &=& C_{n,0} + C_{n,\beta}\beta + C_{n,p}\frac{b}{2V_a}p + C_{n,r}\frac{b}{2V_a}r + C_{n,\delta_a}\delta_a + C_{n,\delta_r}\delta_r \IEEEyessubnumber
\end{IEEEeqnarray}
%
\begin{lstlisting}
	C_l = C_l_0 + C_l_beta*beta + C_l_p*b*0.5/V_a*p + C_l_r*b*0.5/V_a*r + C_ldelta_a*delta_a + C_ldelta_r*delta_r 
	C_m = C_m_0 + C_m_alpha*alpha + C_m_q*c*0.5/V_a*q + C_m_delta_e*delta_e 
	C_n = C_n_0 + C_n_beta*beta + C_n_p*b*0.5/V_a*p + C_n_r*b*0.5/V_a*r + C_n_delta_a*delta_a + C_n_delta_r*delta_r 
\end{lstlisting}
%
The coefficients which correspond to state variables ($C_{D,\alpha}$, $C_{D,q}$, $C_{Y,\beta}$, $C_{Y,p}$, $C_{Y,r}$, $C_{L,\alpha}$, $C_{L,q}$, $C_{l,\beta}$,
$C_{l,p}$, $C_{l,r}$, $C_{m,\alpha}$, $C_{m,q}$, $C_{n,\beta}$, $C_{n,p}$, $C_{n,r}$) in most cases have values such that the resulting forces and moments tend to revert the aircraft to straight and level flight. Hence they are called \emph{stability derivatives}.

On the other hand, coefficients which correspond to the input variables ($C_{D\delta e}$, $C_{D\delta a}$, $ C_{Y\delta r}$,
 $C_{L,\delta_e}$, $C_{l,\delta_a}$, $C_{l,\delta_r}$, $C_{m,\delta_e}$, $C_{n,\delta_a}$, $C_{n,\delta_r}$) describe how the aircraft responds to control inputs and are hence called \emph{control derivatives}.

\todo[inline]{Discuss about $C_{L\alpha}$ shape}

Not all of the above derivatives need to be used in the formulation of the aerodynamic model of a plane. Some of them might be statistically insignificant for certain airframes \cite{Klein2006}. On the other hand, longitudinal quantities usually do not incorporate coefficients of lateral quantities and vice versa. This is in-line with the effort of decoupling the aircraft dynamics into a longitudinal and lateral plane, as much as possible, but there also are physical reasons, stemming from the aircraft geometry.
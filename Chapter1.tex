\chapter{Structural Modeling for Fault Diagnosis}

Fault Diagnosis, or more extensively Fault Detection, Isolation and Identification is gaining momentum in the area of Unmanned Aerial Vehicles (UAVs) lately. Old methods, used in the chemical industry in the past decades, find new applications in the UAV industry all the while new problems appear, mainly due to the faster dynamics and larger uncertainty of the system.
Fault diagnosis aims in providing methods which allow the system to assess possible damage, deviation from the nominal model and unmodeled disturbances, thus constituting a formidable health-management tool. This is all the more crucial in small-scale UAVs, where hardware component redundancy is not an option, due to cost and weight reasons, and smarter ways to benefit from analytical redundancy are required.

Since computing power has become increasingly cheaper, it is interesting to examine the fault diagnosis problem over the whole system in real-time. However, in order to handle such large complexity efficiently and methodically, a way to formalize fault diagnosis is needed. Structural analysis offers such tools. By representing the system via a structure graph (a bi-partite graph to be exact) fault detectability and isolability analysis can be done formally, without the need for human-designed observers. Mathematical tools such applied on incidence matrices, such as the Dulmage-Mendelson decomposition, provide the system designer with much-needed overview and hints towards where residuals can be extracted from and where additional sensor should be placed.

However, before such an anaysis can take place, the system structural graph needs to be constructed. To that goal, over the next chapters, the system model equations will be presented, along with sensor models and proposed fault insertion points. This description is focused mostly on modeling systems where sensors are not easy to introduce (such as a propulsion health monitor). On the contrary, systems whose faults are reported in great detail, such as GPS outages, will not be modeled.